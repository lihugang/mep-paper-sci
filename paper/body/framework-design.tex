\section{架构设计}
\subsection{技术选型}
\p{
    本论文的核心要点就是要在图片上编辑\textcolor{red}{公式},那么最重要的功能必然是进行公式的渲染。目前,在计算机中进行公式的渲染大多使用
    \TeX
    \refTo{tex-introduction}
    。
}
\p{
    \TeX
    是一套排版系统,由计算机科学家、斯坦福大学教授\, Donald Knuth \, 设计和编写的,于1978年首次发布,这是最复杂的数字印刷系统之一。
    \TeX
    被广泛应用于学术界,尤其是数学、计算机科学、经济学、工程学、物理学等等。在
    \TeX
    的基础上,衍生出了许多封装更优雅的排版系统,例如编写这篇论文所用的排版系统\, Xe
    \LaTeX 就是基于
    \TeX
    封装的。
}
\p {
    作为中学生,我们没有足够的知识储备和精力去维护一套新的渲染公式的系统,只能选择站在前人的肩膀上进行我们的项目。但是,
    \TeX
    在不同的平台上也有不同的实现,经过仔细筛查,我选用了Web+KaTeX
    \refTo{katex-introduction}
    的方案。从TexLive
    \refTo{texlive-introduction}
    安装的
    \TeX
    过于庞大而且只方便生成pdf,并不方便直接渲染在图片上。而MathJax
    \refTo{mathjax-introduction}
    的速度太慢,可能会对性能造成一定影响。所以我选择了KaTeX,这是一个能在网页上渲染公式的库,它的渲染速度最快,不少支持显示公式的网站也使用了这个库,例如OpenAI的ChatGPT
    \refTo{chatgpt-introduction}。
}
\p {
    同时,使用KaTeX支持在网页上运行,配合将文档元素转成图片的库html2canvas
    \refTo{html2canvas-introduction}
    就可以将KaTeX渲染出的元素转换成图片,方便在原先的图片上叠加图层。另外,使用Web的好处是只需要打开浏览器就可以快速编辑,免去安装的麻烦,更加人性化。如果要发行客户端,只要往上套好Electron
    \refTo{electron-introduction}
    ,做好兼容层就可以。可以一次编写,到处运行。
}