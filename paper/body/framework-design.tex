\section{架构设计}
\subsection{技术选型}
\p{
    本论文的核心要点就是要在图片上编辑\textcolor{red}{公式},那么最重要的功能必然是进行公式的渲染。目前,在计算机中进行公式的渲染大多使用
    \TeX
    \refTo{tex-introduction}
    。
}
\p{
    \TeX
    是一套排版系统,由计算机科学家、斯坦福大学教授\, Donald Knuth \, 设计和编写的,于1978年首次发布,这是最复杂的数字印刷系统之一。
    \TeX
    被广泛应用于学术界,尤其是数学、计算机科学、经济学、工程学、物理学等等。在
    \TeX
    的基础上,衍生出了许多封装更优雅的排版系统,例如编写这篇论文所用的排版系统\, Xe
    \LaTeX 就是基于
    \TeX
    封装的。
}
\p {
    作为中学生,我们没有足够的知识储备和精力去维护一套新的渲染公式的系统,只能选择站在前人的肩膀上进行我们的项目。但是,
    \TeX
    在不同的平台上也有不同的实现,经过仔细筛查,我选用了Web+KaTeX
    \refTo{katex-introduction}
    的方案。从TexLive
    \refTo{texlive-introduction}
    安装的
    \TeX
    过于庞大而且只方便生成pdf,并不方便直接渲染在图片上。而MathJax
    \refTo{mathjax-introduction}
    的速度太慢,可能会对性能造成一定影响。所以我选择了KaTeX,这是一个能在网页上渲染公式的库,它的渲染速度最快,不少支持显示公式的网站也使用了这个库,例如OpenAI的ChatGPT
    \refTo{chatgpt-introduction}。
}
\p {
    同时,使用KaTeX支持在网页上运行,配合将文档元素转成图片的库html2canvas
    \refTo{html2canvas-introduction}
    就可以将KaTeX渲染出的元素转换成图片,方便在原先的图片上叠加图层。另外,使用Web的好处是只需要打开浏览器就可以快速编辑,免去安装的麻烦,更加人性化。如果要发行客户端,只要往上套好Electron
    \refTo{electron-introduction}
    ,做好兼容层就可以。可以一次编写,到处运行。
}

\p {
    计划将编辑器分为四个模块:总体的效果渲染,代码编辑器,当前行效果预览和图片管理。
}

\subsection{渲染系统}
\p{
    公式渲染是这个设计中最核心的部分,我们的设计应该同时具备实用性和易用性,那么“所见即所得”是我们追求的目标之一,在编辑器中应该划分一部分出来区域来渲染预览效果。为了保证预览效果,预览区域的形状大小应该和画布形状大小是相似的,即预览的宽高比和原始画布的宽高比相等。用数学的方法来写,就是
    \begin{equation}
        \frac{
            width_{canvas}
        }{
            height_{canvas}
        }
        =
        \frac{
            width_{raw}
        }{
            height_{raw}
        }\label{image_ratio}
    \end{equation}
    这样能保证预览不会变形。\\
    为了方便计算,我们将上式变形为
    \begin{equation}
        ratio = \frac{
            width_{canvas}
        }{
            width_{raw}
        } = \frac{
            height_{canvas}
        }{
            height_{raw}
        }\label{image-ratio-2}
    \end{equation}
}
\footnote{$ratio$: 比例}
\footnote{$width_{canvas}$:预览区宽度}
\footnote{$height_{canvas}$:预览区高度}
\footnote{$width_{raw}$:原始图像宽度}
\footnote{$height_{raw}$:原始图像高度}
\p{
    前面已经提到一些,文本渲染的编译链如下 \\ \\
}
\begin{tikzpicture}[node distance=120pt]
\node[draw, rounded corners] (source) {源文本};
\node[draw, rounded corners, right=of source] (element) {HTML元素};
\node[draw, rounded corners, right=80pt of element] (element-with-math) {已渲染公式的元素};
\node[draw, rounded corners, below=20pt of element-with-math] (image) {图层};
\node[draw, rounded corners, below=20pt of source] (result) {预览图片};

\graph{
    (source) -> ["正则表达式切割tokens"above](element) -> ["KaTeX"above](element-with-math) -> ["html2canvas"left](image) -> ["CanvasRenderingContext2D.drawImage()"above](result);
};
\end{tikzpicture}

\p{
    如果想添加其他图片、线条、形状的话可以在图层的地方合并生成预览图。
}

\subsection{用户输入}
\p{
    在 Microsoft Word 中,输入公式只能靠鼠标单击键入,虽说 Word 提供了一点
    \LaTeX
    支持,但是兼容性仍然不是很好,无法做到在正统的
    \LaTeX
    中那么流畅,使用 Word 的工作效率未免有些低。所以,我们仿照
    \LaTeX
    的设计,设计了一种脚本,也就是通过代码来帮助绘图。键入代码需要文本编辑器,在这里我选择用 Monaco Editor
    \refTo{monaco-editor-introduction}
    。
}
\p{
    Monaco Editor 是微软的开源项目之一,是一个基于浏览器的代码编辑器。微软著名的代码编辑器,号称重新定义代码编辑的,Visual Studio Code
    \refTo{vscode-introduction}
    就是脱胎自 Monaco Editor。
}
\p{
    设计每帧从 Monaco Editor 中读取代码内容进行分析,生成错误或警告在编辑器中提示,识别当前行内容告诉预览模块显示什么内容,这样可以做到实时的效果用户体验更好。
}

\subsection{预览}
\p{
    因为全部代码绘图有些许延时(比如说将元素转换成图片需要几秒时间),并不能很方便快捷的查看当前行的设置(字体/颜色/图片)会有什么效果,所以专门设置了预览区,省去合并图层的步骤更加节省时间。
}

\subsection{图片管理}
\p{
    有时候需要从外部导入图片(比如说几何图形或者校徽),设计了一个区域来管理和预览图片,默认缩放到容器高度$50\%$,这样可以在保证能看清图片的情况下堆放更多的图片。
}

\subsection{页面布局}
\p{
    计划将编辑器的左上作为输入、编辑代码的区域,右上为总体效果的预览区,左下为单行效果的预览区,右下为图片管理区。
}
\subsubsection{布局大小计算}
\p{
    根据\eqref{image_ratio},要保证右上的预览与画布尺寸等比缩放。不妨设页面宽度为$width_{container}$,高度为$height_{container}$\,\,。
    则有
    \begin{equation}
        width_{editor} = width_{container} - width_{canvas}\label{editor-width-1}
    \end{equation}
    \begin{equation}
        height_{editor} = height_{canvas}\label{editor-height-1}
    \end{equation}
    \begin{equation}
        width_{image} = width_{canvas}\label{image-width-1}
    \end{equation}
    \begin{equation}
        height_{image} = height_{container} - height_{canvas}\label{image-height-1}
    \end{equation}
    \begin{equation}
        width_{preview} = width_{container} - width_{canvas}\label{preview-width-1}
    \end{equation}
    \begin{equation}
        height_{preview} = height_{container} - height_{canvas}\label{preview-height-1}
    \end{equation}
    对应关系如下表
    \begin{table}[h!]
        \begin{center}
            \caption{编辑器布局变量名对应表}
            \begin{tabular}{l|c|r}
                \textbf{名称}           & \textbf{下标} & \textbf{内容}    \\
                \hline
                \multirow{5}{*}{width}  & container     & 容器(页面)宽度 \\
                                        & canvas        & 画布宽度         \\
                                        & editor        & 代码编辑器宽度   \\
                                        & preview       & 单行预览区宽度   \\
                                        & image         & 图片管理区宽度   \\
                \hline
                \multirow{5}{*}{height} & container     & 容器(页面)高度 \\
                                        & canvas        & 画布高度         \\
                                        & editor        & 代码编辑器高度   \\
                                        & preview       & 单行预览区高度   \\
                                        & image         & 图片管理区高度   \\
            \end{tabular}
        \end{center}
    \end{table}
}

\subsubsection{画布缩放}
\p{
    当画布原始尺寸大于预览区大小时,我们需要对其缩放。不妨设画布原始宽度为$width_{raw}$,原始高度为$height_{raw}$。在计算时,首先要判断是横屏还是竖屏,如果是横屏,那么要尽可能在水平方向上缩放到最大,如果是竖屏,那么要尽可能在竖直方向上缩放到最大,能保证画布始终是最大且不会变形的。
}
\p{
    需要注意的一点是,需要对预览区大小做一定限制,以防止其占用区域过大影响到代码编辑器和图像管理区不能正常使用。
    \begin{table}[h!]
        \begin{center}
            \caption{各分区最小最大尺寸}
            \begin{tabular}{l|c|c|r}
                \textbf{名称}  & \textbf{最小尺寸(宽度 $\times$ 高度)} & \textbf{最大尺寸(宽度 $\times$ 高度)}                        \\
                \hline
                代码编辑器     & $40\% \times 80$                      & $(width_{container} - 240) \times 80\%$                      \\
                画布           & $240 \times 80$                       & $60\% \times 80\%$                                           \\
                单行效果预览区 & $40\% \times 20\%$                    & $(width_{container} - 240) \times (height_{container} - 80)$ \\
                图片管理区     & $240 \times 20\%$                     & $60\% \times (height_{container} - 80)$                      \\
            \end{tabular}
        \end{center}
    \end{table}
    我们为左侧代码编辑器和效果预览区预留了至少$40\% \times width_{container}$的宽度,为下方的效果预览和图片管理预留了至少$20\% \times height_{container}$的高度。
}
\p{
    如果页面过小的时候,画布可能会被缩放的很小,所以要对此限制一下最小的尺寸为$240 \times 80$。
}
\p{
    如果是横屏,先计算宽度。
}
\begin{equation}
    width_{canvas} = min(width_{raw}, width_{container} \times 60\%)
\end{equation}
\p{
    如果原始尺寸很小就不必缩放了,否则预览的字体大小图片定位可能和最终的渲染图有偏差。\\
    根据
    \eqref{image-ratio-2}
    那么缩放比例就为
}
\begin{equation}
    ratio = \frac{width_{canvas}}{width_{raw}}
\end{equation}
\p{
    画布的高度就是
}
\begin{equation}
    height_{canvas} = height_{raw} * ratio
\end{equation}
\p{
    注意到,$height_{canvas} \ge 80\% \times height_{container}$是可能的,会对页面布局造成影响,所以在算完后还有特判一下高度是否大于$80\%$。
}
\paragraph{\\}
\p{
    竖屏就先计算高度。
}
\begin{equation}
    height_{canvas} = min(height_{raw}, height_{container} \times 80\%)
\end{equation}
\p{
    根据
    \eqref{image-ratio-2}
    缩放比例为
}
\begin{equation}
    ratio = \frac{height_{canvas}}{height_{raw}}
\end{equation}
\p{
    宽度是
}
\begin{equation}
    width_{canvas} = width_{raw} * ratio
\end{equation}
\p{
    同样需要特判下宽度是否大于$60\%$
}
\p{
    结合上述的计算过程,这是最终在工程上的源代码。
}
.\\
\begin{lstlisting}[language=C]
    if (rawCanvasSize.width >= rawCanvasSize.height) {
        // landscape
        // fill width first
        transferCanvasSize.width = Math.min(containerWidth * 0.6 /* leave at least 40% space for editor */, rawCanvasSize.width);
        transferCanvasSize.height = ~~(rawCanvasSize.height * transferCanvasSize.width / rawCanvasSize.width); // scale in the same ratio
        if (transferCanvasSize.height > containerHeight * 0.8 /* leave at least 20% space for image manager and render preview */) {
            // oh, it's too high
            // we choose filling height
            transferCanvasSize.height = containerHeight * 0.8;
            transferCanvasSize.width = ~~(rawCanvasSize.width * transferCanvasSize.height / rawCanvasSize.height); // scale in the same ratio
        }
        if (transferCanvasSize.height < 80) {
            transferCanvasSize.height = 80;
        }
        if (transferCanvasSize.width < 240) {
            transferCanvasSize.width = 240;
        }
    } else {
        // portrait
        // fill height first
        transferCanvasSize.height = Math.min(containerHeight * 0.8, rawCanvasSize.height);
        transferCanvasSize.width = ~~(rawCanvasSize.width * transferCanvasSize.height / rawCanvasSize.height);
        if (transferCanvasSize.width > containerWidth * 0.6) {
            transferCanvasSize.width = containerWidth * 0.6;
            transferCanvasSize.height = ~~(rawCanvasSize.height * transferCanvasSize.width / rawCanvasSize.width);
        }
        if (transferCanvasSize.height < 80) {
            transferCanvasSize.height = 80;
        }
        if (transferCanvasSize.width < 240) {
            transferCanvasSize.width = 240;
        }
    }
\end{lstlisting}