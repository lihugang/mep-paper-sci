\section{架构设计}
\subsection{技术选型}
\p{
    本论文的核心要点就是要在图片上编辑\textcolor{red}{公式},那么最重要的功能必然是进行公式的渲染。目前,在计算机中进行公式的渲染大多使用
    \TeX
    \refTo{tex-introduction}
    。
}
\p{
    \TeX
    是一套排版系统,由计算机科学家、斯坦福大学教授\, Donald Knuth \, 设计和编写的,于1978年首次发布,这是最复杂的数字印刷系统之一。
    \TeX
    被广泛应用于学术界,尤其是数学、计算机科学、经济学、工程学、物理学等等。在
    \TeX
    的基础上,衍生出了许多封装更优雅的排版系统,例如编写这篇论文所用的排版系统\, Xe
    \LaTeX 就是基于
    \TeX
    封装的。
}
\p {
    作为中学生,我们没有足够的知识储备和精力去维护一套新的渲染公式的系统,只能选择站在前人的肩膀上进行我们的项目。但是,
    \TeX
    在不同的平台上也有不同的实现,经过仔细筛查,我选用了Web+KaTeX
    \refTo{katex-introduction}
    的方案。从TexLive
    \refTo{texlive-introduction}
    安装的
    \TeX
    过于庞大而且只方便生成pdf,并不方便直接渲染在图片上。而MathJax
    \refTo{mathjax-introduction}
    的速度太慢,可能会对性能造成一定影响。所以我选择了KaTeX,这是一个能在网页上渲染公式的库,它的渲染速度最快,不少支持显示公式的网站也使用了这个库,例如OpenAI的ChatGPT
    \refTo{chatgpt-introduction}。
}
\p {
    同时,使用KaTeX支持在网页上运行,配合将文档元素转成图片的库html2canvas
    \refTo{html2canvas-introduction}
    就可以将KaTeX渲染出的元素转换成图片,方便在原先的图片上叠加图层。另外,使用Web的好处是只需要打开浏览器就可以快速编辑,免去安装的麻烦,更加人性化。如果要发行客户端,只要往上套好Electron
    \refTo{electron-introduction}
    ,做好兼容层就可以。可以一次编写,到处运行。
}

\subsection{渲染系统}
\p{
    公式渲染是这个设计中最核心的部分,我们的设计应该同时具备实用性和易用性,那么“所见即所得”是我们追求的目标之一,在编辑器中应该划分一部分出来区域来渲染预览效果。为了保证预览效果,预览区域的形状大小应该和图片形状大小是相似的,即预览的宽高比和原始图片的宽高比相等。用数学的方法来写,就是
    \begin{equation}
        ratio = 
        \frac{
            width_{preview}
        }{
            height_{preview}
        }
        =
        \displaystyle\frac{
            width_{origin}
        }{
            height_{origin}
        }\label{image_ratio}
    \end{equation}
    这样能保证预览不会变形。
}
\p{
    前面已经提到一些,文本渲染的编译链如下 \\
}
\begin{tikzpicture}[node distance=120pt]
\node[draw, rounded corners] (source) {源文本};
\node[draw, rounded corners, right=of source] (element) {HTML元素};
\node[draw, rounded corners, right=80pt of element] (element-with-math) {已渲染公式的元素};
\node[draw, rounded corners, below=20pt of element-with-math] (image) {图层};
\node[draw, rounded corners, below=20pt of source] (result) {预览图片};

\graph{
    (source) -> ["正则表达式切割tokens"above](element) -> ["KaTeX"above](element-with-math) -> ["html2canvas"left](image) -> ["CanvasRenderingContext2D.drawImage()"above](result);
};
\end{tikzpicture}

\p{
    如果想添加其他图片、线条、形状的话可以在图层的地方合并生成预览图。
}

\subsection{用户输入}