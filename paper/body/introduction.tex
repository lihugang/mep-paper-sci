\section{绪论}
\subsection[研究背景]{研究背景}
\p{
    如今,有许多工具能帮助人们在电脑上编辑公式。例如,Microsoft Word \refTo{microsoft-word-introduction}
    ,\LaTeX
    \refTo{latex-introduction}
    等。但是,这些工具都只能编辑或生成文档,不能很方便的对图片进行操作。到目前为止,在图片上编辑公式最方便的办法就是使用多数图片编辑器中自带的画笔,结合手写板进行手写,但是,手写板毕竟只是字迹,写出来的公式带有个人的笔风,并不像Word和\LaTeX
    所生成的那么标准,并且容易写错和被人误认。而且,没有手写板时,只用鼠标写出来的字迹不堪入目。
}

\subsection[研究目的与意义]{研究目的与意义}
\p{
    对此,我们意图发明一个能快速在图片上编辑公式的编辑器,能帮助广大教师学生更好的批改作业、分享题目思路解答,提高效率,节省时间。
}

\subsection[创新点]{创新点}
\p{
    在Microsoft Word中,编辑公式需要用鼠标去点击,虽说操作起来比较简单,上手难度低,但是降低了工作的效率。而在排版系统
    \LaTeX
    中,公式是用键盘键入的,就像这样\quad \lstinline[language=bash]{x=\\frac{-b \\pm \\sqrt{b^2-4ac}}{2a}},渲染效果是这样的$x=\frac{-b \pm \sqrt{b^2-4ac}}{2a}$,使用键盘进行输入大大提升了工作效率。仿照
    \LaTeX
    ,我将这个图片上的公式编辑器也设计成了像
    \LaTeX
    一样的命令式,上手程度远低于
    \LaTeX
    ,只需花五分钟学习就可以轻松上手,编辑效率提升许多。
}