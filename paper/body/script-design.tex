\subsection{命令设计}
\subsubsection{画布元数据}
\p{
    在画布中我们需要设置一些基本信息,例如画布尺寸,字体颜色,字体大小,用什么字体等等。如果在画布外面单独弄一块区域出来设置,那么在调整的时候还是得需要鼠标,会对工作效率有一定影响。为此,设置画布的基本信息应该也在命令系统中提供。既然是设置信息,不妨就将命令名称称为\lstinline|set|,那么就设置字体就是\lstinline|set font|,设置颜色\lstinline|set color|,设置字体大小\lstinline|size|,设置画布尺寸\lstinline|set canvas|。可以得到以下
}
DFA\footnote{确定有限自动状态机}
\paragraph{\\\\}
\begin{tikzpicture}[node distance=50pt]
    \node[draw, rounded corners, fill=green] (set) {set};
    \node[draw, rounded corners, right=of set, fill=green] (canvas) {canvas};
    \node[draw, rounded corners, right=of canvas, fill=green] (cv-width) {[宽度]};
    \node[draw, rounded corners, right=of cv-width, fill=green] (cv-height) {[高度]};
    \node[draw, rounded corners, below=20pt of cv-width, fill=red] (range-error) {值域错误};

    \node[draw, rounded corners, above=of canvas, fill=green] (color) {color};
    \node[draw, rounded corners, right=160pt of color, fill=green] (color-value) {[颜色,十六进制颜色编码,6位]};
    \node[draw, rounded corners, below=20pt of color-value, fill=yellow] (color-invalid-range-warning) {未识别的颜色编码};

    \node[draw, rounded corners, below=60pt of canvas, fill=green] (size) {size};
    \node[draw, rounded corners, right=of size, fill=green] (size-value) {[字体大小]};

    \node[draw, rounded corners, below=of size, fill=green] (font) {font};
    \node[draw, rounded corners, right=of font, fill=green] (font-name) {[字体名称]};
    \node[draw, rounded corners, above=20pt of font-name, fill=yellow] (invalid-font-name) {未找到字体};

    \node[draw, rounded corners, below=of set, fill=red] (unknown-ast-symbol) {未知标识符};

    \graph{
    (set) -> (canvas) -> ["$\in N*$"](cv-width) -> ["$\in N*$"](cv-height);
    (canvas) -> ["$\notin N*$"](range-error);
    (cv-width) -> ["$\notin N*$"](range-error);

    (set) -> (color) -> ["$length = 6$"above,"值是合法的十六进制数"below](color-value);
    (color) -> ["默认"below](color-invalid-range-warning);

    (set) -> (size) -> ["$\in N*$"](size-value);
    (size) -> ["$\notin N*$"](range-error);

    (set) -> (font) -> ["$\in fonts$"](font-name);
    (font) -> ["$\notin fonts$"](invalid-font-name);

    (set) -> ["默认"](unknown-ast-symbol);
    };
\end{tikzpicture}
\footnote{绿色部分是合法的标识符,黄色是警告,红色是错误(会中断编译)}

\p{
    宽度、高度、字体大小都必须是正整数($\in N*$),颜色编码应该为6位十六进制数字,代表RGB,前两位是红色的强度,中间两位是蓝色的强度,后两位是绿色的强度,值域为$[0, 255]$(十进制)或$[00, ff]$(十六进制),值越大表示强度越大,占比越多。
    \begin{table}[h!]
        \begin{center}
            \caption{常见颜色的RGB值}
            \begin{tabular}{l|c|c|}
                \textbf{颜色} & \textbf{十六进制颜色编码} & \textbf{渲染效果}                            \\
                \hline
                白色          & $ffffff$                  & \colorbox[rgb]{1,1,1}{白底黑字}              \\
                黑色          & $000000$                  & \colorbox[rgb]{0,0,0}{\color{white}黑底白字} \\
                红色          & $ff0000$                  & \colorbox[rgb]{1,0,0}{\color{white}红底白字} \\
                绿色          & $00ff00$                  & \colorbox[rgb]{0,1,0}{绿底黑字}              \\
                蓝色          & $0000ff$                  & \colorbox[rgb]{0,0,1}{\color{white}蓝底白字} \\
                黄色          & $ffff00$                  & \colorbox[rgb]{1,1,0}{黄底黑字}              \\
                青色          & $00ffff$                  & \colorbox[rgb]{0,1,1}{青底黑字}              \\
                紫色          & $ff00ff$                  & \colorbox[rgb]{1,0,1}{\color{white}紫底白字} \\
            \end{tabular}
        \end{center}
    \end{table}
}
\subsubsection{定位方式}
\p{
    关于元素如何在画布中定位,我设计了两种定位方式。第一种是$abs$,无论画布大小如何变化,元素始终会渲染在给出的坐标位置。第二种是$rwd$,响应式布局,需要给定一个纯小数(百分比),具体的渲染坐标会随着画布的大小变化而变化,但与各边保持的比例始终相同。下面是计算公式 \\
}
\begin{equation}
    x_{transformed} = x_{percentage} \times canvas_{width}\label{transform-rwd-x}
\end{equation}
\begin{equation}
    y_{transformed} = y_{percentage} \times canvas_{height}\label{transform-rwd-y}
\end{equation}
\footnote{$transformed$: 已经转换好的绝对坐标}
\footnote{$percentage$: 元素在页面中位置的百分比}
\subsubsection{文本渲染}
\p{
    结合元素的两种不同定位方式,得到以下DFA\\\\
}
\begin{tikzpicture}[node distance=120pt]
    \node[draw, rounded corners, fill=green] (text) {text};

    \node[draw, rounded corners, left=of text, fill=green] (abs) {abs};
    \node[draw, rounded corners, right=of text, fill=green] (rwd) {rwd};

    \node[draw, rounded corners, below=of text, fill=red] (unknown-symbol) {未知标识符};
    \node[draw, rounded corners, below=40pt of unknown-symbol, fill=yellow] (range-warning) {值域错误($abs$定位应$\in N*$,$rwd$定位应$\in [0, 1]$)};

    \node[draw, rounded corners, below=of abs, fill=green] (abs-width) {[宽度]};
    \node[draw, rounded corners, below=of abs-width, fill=green] (abs-height) {[高度]};
    \node[draw, rounded corners, below=of rwd, fill=green] (rwd-width) {[宽度]};
    \node[draw, rounded corners, below=of rwd-width, fill=green] (rwd-height) {[高度]};

    \node[draw, rounded corners, below=of range-warning, fill=green] (texts) {[文本]};

    \graph{
    (text) -> (abs) -> ["$\in N*$"](abs-width) -> ["$\in N*$"](abs-height) -> (texts);
    (text) -> (rwd) -> ["$\in [0.0, 1.0]$"](rwd-width) -> ["$\in [0.0, 1.0]$"](rwd-height) -> (texts);

    (text) -> ["默认"](unknown-symbol);

    (abs) -> ["$\notin N*$"](range-warning);
    (rwd) -> ["$\notin N*$"](range-warning);
    (abs) -> ["不是数字"](unknown-symbol);
    (rwd) -> ["不是数字"above](unknown-symbol);

    (abs-width) -> ["$\notin N*$"](range-warning);
    (rwd-width) -> ["$\notin N*$"](range-warning);
    (abs-width) -> ["不是数字"](unknown-symbol);
    (rwd-width) -> ["不是数字"](unknown-symbol);
    };
\end{tikzpicture}
\subsubsection{画线}
\p{
    从$(sx, sy)$画线条到$(dx, dy)$,线条粗细受\lstinline|set size|语句影响 \\ \\
}
\begin{tikzpicture}[node distance=60pt]
    \node[draw, rounded corners, fill=green] (draw) {draw};
    \node[draw, rounded corners, right=of draw, fill=green] (from) {from};
    \node[draw, rounded corners, right=of from, fill=green] (sx) {开始位置/横坐标};
    \node[draw, rounded corners, right=of sx, fill=green] (sy) {开始位置/纵坐标};
    \node[draw, rounded corners, below=of sy, fill=green] (to) {to};
    \node[draw, rounded corners, left=of to, fill=green] (dx) {结束位置/横坐标};
    \node[draw, rounded corners, left=of dx, fill=green] (dy) {结束位置/纵坐标};

    \node[draw, rounded corners, above=150pt of from, fill=red] (unknown-symbol) {未知标识符};
    \node[draw, rounded corners, below=120pt of dx, fill=yellow] (range-warning) {值域错误};

    \graph{
    (draw) -> (from) -> ["$\in N*$"](sx) -> ["$\in N*$"](sy) -> (to) -> ["$\in N*$"](dx) -> ["$\in N*$"](dy);

    (draw) -> ["默认"left](unknown-symbol);
    (sy) -> ["默认"right](unknown-symbol);

    (from) -> ["不是数字"](unknown-symbol);
    (sx) -> ["不是数字"](unknown-symbol);
    (to) -> ["不是数字"](unknown-symbol);
    (dx) -> ["不是数字"](unknown-symbol);

    (from) -> ["$\notin N*$"](range-warning);
    (sx) -> ["$\notin N*$"](range-warning);
    (to) -> ["$\notin N*$"](range-warning);
    (dx) -> ["$\notin N*$"](range-warning);
    };
\end{tikzpicture}
\p{
    需要注意一点,这里的坐标都是在$abs$布局下的,对于$rwd$布局的支持作为可以改进的部分。
}
\subsubsection{添加图片}
\p{
    有时候也需要添加一些几何图形或其他图片到画布上,命令系统也提供了支持。添加图片包括三个部分:
}
\begin{itemize}
    \item 图片ID
    \item 图片位置
    \item 图片尺寸
\end{itemize}
\p{
    命令格式如下
}
\begin{math}
    image
    \begin{matrix}
        \underbrace{
            abs/rwd \begin{matrix}
                        ID \\ \overbrace{[sha256]}
                    \end{matrix} at [x] [y]
        } \\ location
    \end{matrix}
    \begin{matrix}
        \underbrace{resize [width] [height]} \\ size
    \end{matrix}
\end{math}
\footnote{$image: $图片命令}
\footnote{$ID: $图片ID}
\footnote{$location: $图片定位}
\footnote{$size: $图片尺寸}
\p{
    根据上面的命令格式,可以生成一张很大的DFA图:\\\\
}
\begin{tikzpicture}[node distance=80pt]
    \node[draw, rounded corners, fill=green] (image) {image};

    \node[draw, rounded corners, left=140pt of image, fill=green] (loc-abs) {abs};
    \node[draw, rounded corners, right=140pt of image, fill=green] (loc-rwd) {rwd};

    \node[draw, rounded corners, below=20pt of image, fill=green] (id) {[图片ID]};

    \node[draw, rounded corners, below=20pt of id, fill=green] (at) {at};

    \node[draw, rounded corners, left=140pt of at, fill=green] (abs-x) {[横坐标]};
    \node[draw, rounded corners, below=200pt of abs-x, fill=green] (abs-y) {[纵坐标]};
    \node[draw, rounded corners, right=140pt of at, fill=green] (rwd-x) {[横坐标]};
    \node[draw, rounded corners, below=200pt of rwd-x, fill=green] (rwd-y) {[纵坐标]};

    \node[draw, rounded corners, below=of at, fill=blue] (s1) {};
    \node[draw, rounded corners, left=40pt of s1, fill=red] (unknown-symbol) {未知标识符};

    \node[draw, rounded corners, below=160pt of s1, fill=green] (resize) {resize};

    \node[draw, rounded corners, below=of resize, fill=blue] (s2) {};
    \node[draw, rounded corners, right=40pt of s2, fill=orange] (range-warning) {值域错误};

    \node[draw, rounded corners, left=140pt of resize, fill=green] (resize-abs) {abs};
    \node[draw, rounded corners, right=140pt of resize, fill=green] (resize-rwd) {rwd};

    \node[draw, rounded corners, below=of resize-abs, fill=green] (abs-w) {[宽度]};
    \node[draw, rounded corners, below=of abs-w, fill=green] (abs-h) {[高度]};
    \node[draw, rounded corners, below=of resize-rwd, fill=green] (rwd-w) {[宽度]};
    \node[draw, rounded corners, below=of rwd-w, fill=green] (rwd-h) {[高度]};

    \graph{
    (image) -> [very thick,green](loc-abs) -> [very thick,green](id) -> ["$\in images$",very thick,green](at) -> ["定位方式$=abs$且$\in N*$",green,very thick](abs-x) -> ["$\in N*$",green,very thick](abs-y) -> [green,very thick](resize) -> [green,very thick](resize-abs) -> ["$\in N*$",green,very thick](abs-w) -> ["$\in N*$",green,very thick](abs-h);
    (image) -> [green,very thick](loc-rwd) -> [green,very thick](id) -> ["$\in images$",very thick,green](at) -> ["定位方式$=rwd$且$\in [0,1]$",green,very thick](rwd-x) -> ["$\in [0,1]$",green,very thick](rwd-y) -> [green,very thick](resize) -> [green,very thick](resize-rwd) -> ["$\in [0,1]$",green,very thick](rwd-w) -> ["$\in [0,1]$",green,very thick](rwd-h);

    (id) -> ["$\notin images$",orange](range-warning);

    (image) -> ["默认",red,dashed](unknown-symbol);
    (at) -> ["默认",red,dashed](unknown-symbol);
    (resize) -> ["默认",red,dashed](unknown-symbol);

    (abs-x) -> ["不是数字",red,dashed](unknown-symbol);
    (abs-y) -> ["不是数字",red,dashed](unknown-symbol);
    (abs-w) -> ["不是数字",red,dashed](unknown-symbol);
    (abs-h) -> ["不是数字",red,dashed](unknown-symbol);
    (rwd-x) -> ["不是数字",red,dashed](unknown-symbol);
    (rwd-y) -> ["不是数字",red,dashed](unknown-symbol);
    (rwd-w) -> ["不是数字",red,dashed](unknown-symbol);
    (rwd-h) -> ["不是数字",red,dashed](unknown-symbol);

    (abs-x) -> ["$\notin N*$",orange,dotted,thick](range-warning);
    (abs-y) -> ["$\notin N*$",orange,dotted,thick](range-warning);
    (abs-w) -> ["$\notin N*$",orange,dotted,thick](range-warning);
    (abs-h) -> ["$\notin N*$",orange,dotted,thick](range-warning);
    (rwd-x) -> ["$\notin [0, 1]$",orange,dotted,thick](range-warning);
    (rwd-y) -> ["$\notin [0, 1]$",orange,dotted,thick](range-warning);
    (rwd-w) -> ["$\notin [0, 1]$",orange,dotted,thick](range-warning);
    (rwd-h) -> ["$\notin [0, 1]$",orange,dotted,thick](range-warning);
    };
\end{tikzpicture}
\subsubsection{\LaTeX
    宏}
\p{
    公式的环境是\LaTeX
    的数学环境,有时候公式中有大量重复的命令,输入繁琐不方便,就可以设置一个简单的宏来代替。比如用\\
}
\begin{lstlisting}
    \root-1o2
\end{lstlisting}
\p{
    来代替\\
}
\begin{lstlisting}
    x=\frac{-b\pm\sqrt{b^2-4ac}}{2a}
\end{lstlisting}
\p{
    上面例子的命令就是\\
}
\begin{lstlisting}
    macro \root-1o2 x=\frac{-b\pm\sqrt{b^2-4ac}}{2a}
\end{lstlisting}
\p{
    命令定义为:\lstinline|macro 宏名 宏值|
}