\section{命令系统}
\subsection{引言}
\p{
    像 \LaTeX
    一样,我们需要提供一个比较简便的命令系统(注意,这是图灵不完备的)来使用户控制画布的排版、图像的添加、文字的渲染等问题。为了方便使用,命令应该像这样:\lstinline|draw from 30 30 to 300 300|,比较贴切自然语言,使得用户学习起来更方便。另外,以空格作为分隔符,在处理这些命令的时候也比较方便,性能也比较高。
}

\subsection{命令设计}
\subsubsection{画布元数据}
\p{
    在画布中我们需要设置一些基本信息,例如画布尺寸,字体颜色,字体大小,用什么字体等等。如果在画布外面单独弄一块区域出来设置,那么在调整的时候还是得需要鼠标,会对工作效率有一定影响。为此,设置画布的基本信息应该也在命令系统中提供。既然是设置信息,不妨就将命令名称称为\lstinline|set|,那么就设置字体就是\lstinline|set font|,设置颜色\lstinline|set color|,设置字体大小\lstinline|size|,设置画布尺寸\lstinline|set canvas|。可以得到以下抽象语法树。\\ \\
}
\begin{tikzpicture}[node distance=50pt]
    \node[draw, rounded corners, fill=green] (set) {set};
    \node[draw, rounded corners, right=of set, fill=green] (canvas) {canvas};
    \node[draw, rounded corners, right=of canvas, fill=green] (cv-width) {[宽度]};
    \node[draw, rounded corners, right=of cv-width, fill=green] (cv-height) {[高度]};
    \node[draw, rounded corners, below=20pt of cv-width, fill=red] (range-error) {值域错误};

    \node[draw, rounded corners, above=of canvas, fill=green] (color) {color};
    \node[draw, rounded corners, right=160pt of color, fill=green] (color-value) {[颜色,十六进制颜色编码,6位]};
    \node[draw, rounded corners, below=20pt of color-value, fill=yellow] (color-invalid-range-warning) {未识别的颜色编码};

    \node[draw, rounded corners, below=60pt of canvas, fill=green] (size) {size};
    \node[draw, rounded corners, right=of size, fill=green] (size-value) {[字体大小]};

    \node[draw, rounded corners, below=of size, fill=green] (font) {font};
    \node[draw, rounded corners, right=of font, fill=green] (font-name) {[字体名称]};
    \node[draw, rounded corners, above=20pt of font-name, fill=yellow] (invalid-font-name) {未找到字体};

    \graph{
        (set) -> (canvas) -> ["$\in N*$"](cv-width) -> ["$\in N*$"](cv-height);
        (canvas) -> ["$\notin N*$"](range-error);
        (cv-width) -> ["$\notin N*$"](range-error);

        (set) -> (color) -> ["$length = 6$"above,"value is a valid hexadecimal number"below](color-value);
        (color) -> ["otherwise"below](color-invalid-range-warning);

        (set) -> (size) -> ["$\in N*$"](size-value);
        (size) -> ["$\notin N*$"](range-error);

        (set) -> (font) -> ["$\in fonts$"](font-name);
        (font) -> ["$\notin fonts$"](invalid-font-name);
    };
\end{tikzpicture}
\footnote{绿色部分是合法的标识符,黄色是警告,红色是错误(会中断编译)}

\p{
    宽度、高度、字体大小都必须是正整数($\in N*$),颜色编码应该为6位十六进制数字,代表RGB,前两位是红色的强度,中间两位是蓝色的强度,后两位是绿色的强度,值域为$[0, 255]$(十进制)或$[00, ff]$(十六进制),值越大表示强度越大,占比越多。
    \begin{table}[h!]
        \begin{center}
            \caption{常见颜色的RGB值}
            \begin{tabular}{l|c|c|r}
                \textbf{颜色} & \textbf{十六进制颜色编码} & \textbf{渲染效果} \\
                \hline
                白色 & $ffffff$ & \colorbox[rgb]{1,1,1}{白底黑字} \\
                黑色 & $000000$ & \colorbox[rgb]{0,0,0}{\color{white}黑底白字} \\
                红色 & $ff0000$ & \colorbox[rgb]{1,0,0}{\color{white}红底白字} \\
                绿色 & $00ff00$ & \colorbox[rgb]{0,1,0}{绿底黑字} \\
                蓝色 & $0000ff$ & \colorbox[rgb]{0,0,1}{\color{white}蓝底白字} \\
                黄色 & $ffff00$ & \colorbox[rgb]{1,1,0}{黄底黑字} \\
                青色 & $00ffff$ & \colorbox[rgb]{0,1,1}{青底黑字} \\
                紫色 & $ff00ff$ & \colorbox[rgb]{1,0,1}{\color{white}紫底白字} \\
            \end{tabular}
        \end{center}
    \end{table}
}
